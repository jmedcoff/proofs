\documentclass[12pt]{amsart}

\usepackage{amsthm}
\usepackage{amssymb}
\usepackage{amsmath}
\usepackage{geometry}
\geometry{a4paper}

\newtheorem{thm}{Theorem}
\newtheorem{prop}{Proposition}
\newtheorem{lem}{Lemma}
\newtheorem{cor}{Corollary}
\newtheorem{rem}{Remark}
\newtheorem{exa}{Example}
\newtheorem{clm}{Claim}

\newtheoremstyle{case}{}{}{}{}{}{:}{ }{}
\theoremstyle{case}
\newtheorem{case}{Case}

\title{Continuity of the Sine and the Discriminant}
\author{Jason Medcoff}


\begin{document}
	
	\maketitle
	
	\section{Continuity of Sine}
	
	The following lemma will be used to prove Claim 1.
	\begin{lem}
		\label{ineq}
		$|\sin(x)| \leq |x| \ \forall x \in \mathbb{R}$.
	\end{lem}

	\begin{proof}
		Suppose instead that $|\sin(x)| > |x| \ \forall x$. Then $ -|x| > \sin(x) > |x| $. Consider three cases as follows.
		\begin{case}
			$x=0$. Then we have $0 > \sin(0) > 0$, or $0>0>0$. This is impossible.
		\end{case}
		\begin{case}
			$x>0$. We thus have $-x>\sin(x)>x$, which gives $-x>x$. This contradicts the assumption that $x>0$.
		\end{case}
		\begin{case}
			$x<0$. Then we have $x>\sin(x)>-x$. It follows that $x>-x$, which fails to hold for $x<0$ as assumed.
		\end{case}
	
	Each case gives an impossible mathematical statement. So, $|\sin(x)| > |x| \ \forall x$ is incorrect. Therefore, the original proposition is true.		
	\end{proof}
	
	
	
	
	\begin{clm}
		The function $f : \mathbb{R} \to \mathbb{R}$ such that $f(x) = \sin(x)$ is well defined and continuous everywhere on its domain.
	\end{clm}
	
	\begin{proof}
		Let $x \in (-\infty, \infty)$. We know that for $f$ to be well defined, we must have $\forall x$, there is exactly one $y \in \mathbb{R}$ such that $f(x) = y$. Let $y = \sin(x)$. Then $f(x) = y$. Let $x_1, x_2 \in (-\infty, \infty)$ such that $f(x_1) \neq f(x_2)$. Then $\sin(x_1) \neq \sin(x_2)$. Taking the inverse sine, we have $x_1 \neq x_2$. Therefore, $\forall x \in (-\infty, \infty)$, there exists a unique $y$ such that $\sin(x) = y$. So, $f$ is a well defined function.
		
		For a function $f$ to be continuous, we know that the limit
		$$\lim_{x \to a} f(x) = f(a)$$
		must hold for all $a$ in the domain. Let $a \in (-\infty, \infty)$. We want to show that $\forall \varepsilon > 0, \exists \delta > 0$ such that
		$$ \delta > |x-a| > 0 \implies |f(x) - f(a)| < \varepsilon. $$
		Suppose $\varepsilon$ is given. We want to find $\delta$ such that
		$$ |\sin(x) - \sin(a)| < \varepsilon. $$
		We can see by trigonometric identity that
		$$ |\sin(x) - \sin(a)| = \Big|2 \cos\Big(\frac{x+a}{2}\Big) \sin\Big(\frac{x-a}{2}\Big) \Big|. $$
		Since $\cos(\alpha) \leq 1$ $\forall \alpha \in \mathbb{R}$, we can show that
		$$ \Big|2 \cos\Big(\frac{x+a}{2}\Big) \sin\Big(\frac{x-a}{2}\Big) \Big| \leq 2 \Big| \sin\Big(\frac{x-a}{2}\Big)\Big|. $$
		
		
		We want $ \delta > |x-a| > 0$, but notice that $$|x-a| \geq \Big| \frac{x-a}{2} \Big|, $$
		so we can write
		$$ \delta > \Big| \frac{x-a}{2} \Big| > 0 .$$
		Due to lemma \ref{ineq}, we know
		$$ 2 \Big| \frac{x-a}{2} \Big| \geq 2 \Big| \sin\Big(\frac{x-a}{2}\Big)\Big| $$
		is true, so we will write
		$$ 2 \delta > 2 \Big| \frac{x-a}{2} \Big| \geq 2 \Big| \sin\Big(\frac{x-a}{2}\Big)\Big| . $$
		
		If we take $\delta = \frac{\varepsilon}{2}$, then
		$$ 2 \Big| \sin\Big(\frac{x-a}{2}\Big)\Big| < \varepsilon $$
		holds true. As $\delta$ and $\varepsilon$ do not depend on $x$, this is true over the entire domain $\mathbb{R}$. Therefore, 
		$$\lim_{x \to a} \sin(x) = \sin(a)$$
		is true for all $a \in \mathbb{R}$, and we know that $f$ is continuous.
	\end{proof}

	\section{Quadratic Discriminant}
	
	
	A quadratic polynomial $f(x) = ax^2 + bx + c$ can be factored into $a(x-\alpha_1)(x-\alpha_2)$ for some complex numbers $\alpha_1$ and $\alpha_2$. Define the discriminant of $f(x)$ to be
	$$ \Delta_f := \prod_{i\neq j} (\alpha_i - \alpha_j) = (\alpha_1 - \alpha_2)(\alpha_2 - \alpha_1) .$$	
	
	\begin{clm}
		$\Delta_f$ can be written in terms of $a$, $b$, and $c$ by the expression
		$$ \Delta_f = \frac{4ac - b^2}{a^2} . $$
	\end{clm}
	
	\begin{proof}
		We know that the roots $\alpha_1$ and $\alpha_2$ are given by the general quadratic formula. So let
		$$ \alpha_1 = \frac{-b + \sqrt{b^2 - 4ac}}{2a} $$
		
		and let
		$$ \alpha_2 = \frac{-b - \sqrt{b^2 - 4ac}}{2a} . $$
		
		Then we can write $(\alpha_1 - \alpha_2)(\alpha_2 - \alpha_1)$ as
		$$ \bigg( \frac{(-b + \sqrt{b^2 - 4ac})-(-b - \sqrt{b^2 - 4ac})}{2a} \bigg) \bigg( \frac{(-b - \sqrt{b^2 - 4ac})-(-b + \sqrt{b^2 - 4ac})}{2a} \bigg) . $$
		
		Combining terms gives us
		$$ \bigg( \frac{2\sqrt{b^2 - 4ac}}{2a} \bigg) \bigg( \frac{-2\sqrt{b^2 - 4ac}}{2a} \bigg) . $$
		
		Multiplying, we have
		$$ - \frac{(b^2 - 4ac)}{a^2} = \frac{4ac - b^2}{a^2} . $$
	\end{proof}

	\section{Cubic Discriminant}

	The following lemma will be used to prove Claim 3.

	\begin{lem}
		\label{depressed}
		Given any cubic polynomial $f(x) = ax^3 + bx^2 + cx + d$, let $x = y - \frac{b}{3a}$. $f(x)$ can be written as
		$$ g(y) = y^3 + py + q $$
		with
		$$ p = \frac{3ac-b^2}{3a^2} \ , \ q = \frac{2b^3 - 9abc + 27a^2d}{27a^3} . $$
	\end{lem}
	
	\begin{proof}
		Begin by substituting $y-\frac{b}{3a}$ for $x$. Then we obtain
		$$ a \Big( y-\frac{b}{3a} \Big)^3 + b\Big( y-\frac{b}{3a} \Big)^2 + c \Big( y-\frac{b}{3a} \Big) + d . $$
		Expanding and simplifying, we have
		$$ ay^3 + \Big( c - \frac{b^2}{3a} \Big) y + \Big(d + \frac{2b^3}{27a^2} - \frac{bc}{3a} \Big) , $$
		and we can divide by $a$ and use common denominators to get the form
		$$ y^3 + \Big( \frac{3ac-b^2}{3a^2} \Big) y + \Big( \frac{2b^3 - 9abc + 27a^2d}{27a^3} \Big) . $$
		Substituting $p = \frac{3ac-b^2}{3a^2}$ and $q = \frac{2b^3 - 9abc + 27a^2d}{27a^3}$, we obtain
		$$ y^3 + py + q . $$		
	\end{proof}

%	\begin{lem}
%		Let $f(x) = ax^3 + bx^2 + cx + d$ be a cubic polynomial. Once we have obtained the depressed form $$ y^3 + py + q $$ as described in lemma \ref{depressed}, we can write the solutions to $f(x) = 0$ as
%	\end{lem}
%
%	\begin{proof}
%		We need to solve $y^3 + py + q = 0$. Set $st = \frac{p}{3}$ and $s^3 - t^3 = -q$. Then we have
%		$$ (s-t)^3 + 3st(s-t) - (s^3 - t^3) = 0 . $$
%		which leads to
%		\begin{equation*}
%		\begin{split} (s^3 - 3s^2t + 3st^2 - t^3) - (s^3 - 3s^2t + 3st^2 - t^3) & = 0 \\
%		0 & = 0.
%		\end{split}
%		\end{equation*}
%		
%		So we need to find $s$ and $t$ that satisfy $st = \frac{p}{3}$ and $s^3 - t^3 = -q$. The first equation can be written as $s = \frac{p}{3t}$, and substituting into the second, we obtain
%		$$ \frac{p^3}{27t^3} - t^3 = -q . $$
%		Multiplying by $-t^3$, we have
%		$$ t^6 - qt^3 - \frac{p^3}{27} = 0 . $$
%		Let $u = t^3$. Then we have
%		$$u^2 - qu - \frac{p^3}{27} = 0 , $$
%		a quadratic equation in $u$. From the quadratic formula, we have solutions 
%		$$u = \frac{q}{2} + \sqrt{\frac{(q)^2}{4} + \frac{p^3}{27}} \ , \ u = \frac{q}{2} - \sqrt{\frac{(q)^2}{4} + \frac{p^3}{27}} $$
%		which yields
%		$$t = \sqrt[3]{\frac{q}{2} + \sqrt{\frac{(q)^2}{4} + \frac{p^3}{27}}} \ , \ t = \sqrt[3]{\frac{q}{2} - \sqrt{\frac{(q)^2}{4} + \frac{p^3}{27}}} . $$
%		
%		Recall the previously asserted equality $st = \frac{p}{3}$, which implies $s^3 = \frac{p^3}{27t^3} $. Begin by taking the positive root. We have
%		$$ t^3 = \frac{q}{2} + \sqrt{\frac{(q)^2}{4} + \frac{p^3}{27}} , $$
%		which gives
%		\begin{equation*}
%		\begin{split}
%			s^3 & = \frac{p^3}{27 \Big(\frac{q}{2} + \sqrt{\frac{(q)^2}{4} + \frac{p^3}{27}} \Big)}
%		\end{split}
%		\end{equation*}
%		If we take the negative root, we similarly have
%		$$ s^3 = \Big( \frac{q}{2} - \sqrt{\frac{(q)^2}{4} + \frac{p^3}{27}} \Big) \frac{p^3}{27} . $$
%		
%		Let $t_1$ be a particular solution. Construct a root of unity $z = - \frac{1}{2} + \frac{\sqrt{3}}{2}i$. Then the two other solutions are $t_2 = t_1 z$ and $t_3 = t_1 z^2$.
%		
%		Let $s_1 = \frac{P}{3t_1}$. Since $st = \frac{p}{3}$, the solutions to the depressed form polynomial are $s_1 - t_1$, $s_1 z^2 - t_1 z$, and $s_1 z - t_1 z^2$.
%		
%		
%	
%	\end{proof}
	
	
	
	\begin{clm}
		Let $f(x) = ax^3 + bx^2 + cx + d$ be a cubic polynomial. Once we have obtained the depressed form $$ g(y) = y^3 + py + q $$ as described in lemma \ref{depressed}, the discriminant is given by
		$$\Delta_f = 4 p^3 + 27 q^2 . $$
	\end{clm}

	\begin{proof}
		
		We know that for any cubic polynomial $f(x)$, we can write
		\begin{equation*}
		\begin{split}
		f(x) & = ax^3 + bx^2 + cx + d \\
		     & = a (x-\alpha_1)(x-\alpha_2)(x-\alpha_3)
		\end{split}
		\end{equation*}
		
		if it has roots $\alpha_1$, $\alpha_2$, and $\alpha_3$. Expanding, we obtain
		$$ f(x) = ax^3 + ax^2 (-\alpha_1-\alpha_2-\alpha_3) + ax (\alpha_1\alpha_2+\alpha_1\alpha_3+\alpha_2\alpha_3) + a(-\alpha_1\alpha_2\alpha_3) $$
		
		which gives
		\begin{equation*}
		\begin{split}
		b & = a (-\alpha_1-\alpha_2-\alpha_3) \\
		c & = a (\alpha_1\alpha_2+\alpha_1\alpha_3+\alpha_2\alpha_3) \\
		d & = a (-\alpha_1\alpha_2\alpha_3) .
		\end{split}
		\end{equation*}
		
		Considering the case for $ y^3 + py + q $, we have $a = 1$ and $-\alpha_1-\alpha_2-\alpha_3 = 0$. Then let $\alpha_3 = - \alpha_1 - \alpha_2$. We now can expand as above to obtain
		$$ g(y) = \alpha_1^2\alpha_2 - \alpha_1^2 y + \alpha_1\alpha_2^2 - \alpha_1\alpha_2 y - \alpha_2^2 y + y^3 . $$
		This gives
		$$ g(y) = y^3 + y (-\alpha_1^2 + \alpha_1\alpha_2 - \alpha_2^2) + (\alpha_1^2\alpha_2 + \alpha_1\alpha_2^2) , $$
		from which we may derive
		\begin{equation*}
		\begin{split}
		p & = -\alpha_1^2 - \alpha_1\alpha_2 - \alpha_2^2 \\
		q & = \alpha_1^2\alpha_2 + \alpha_1\alpha_2^2 .
		\end{split}
		\end{equation*}
		
		For the cubic, we have the discriminant
		$$ \Delta_f := \prod_{i\neq j} (\alpha_i - \alpha_j) = (\alpha_1 - \alpha_2)(\alpha_2 - \alpha_1)(\alpha_3 - \alpha_2)(\alpha_2 - \alpha_3)(\alpha_1 - \alpha_3)(\alpha_3 - \alpha_1) $$
		
		and since $\alpha_3 = - \alpha_1 - \alpha_2$, we can make the substitution
		$$ \Delta_f = (\alpha_1 - \alpha_2)(\alpha_2 - \alpha_1)((-\alpha_1-\alpha_2) - \alpha_2)(\alpha_2 - (-\alpha_1-\alpha_2))(\alpha_1 - (-\alpha_1-\alpha_2))((-\alpha_1-\alpha_2) - \alpha_1) . $$
		
		Expanding yields
		$$ \Delta_f = -4 \alpha_1^6 - 12 \alpha_1^5 \alpha_2 + 3 \alpha_1^4 \alpha_2^2 + 26 \alpha_1^3 \alpha_2^3 + 3 \alpha_1^2 \alpha_2^4 - 12 \alpha_1 \alpha_2^5 - 4 \alpha_2^6 . $$
		
		Consider the quantity
		$$ 4 p^3 + 27 q^2 . $$
		
		Using the above expressions for $p$ and $q$, we have
		\begin{equation*}
		\begin{split}
		4 p^3 + 27 q^2 & = 4 (-\alpha_1^2 - \alpha_1\alpha_2 - \alpha_2^2)^3 + 27 (\alpha_1^2\alpha_2 + \alpha_1\alpha_2^2)^2 .\\
		\end{split}
		\end{equation*}
		
		Expanding this expression gives
		$$ -4 \alpha_1^6 - 12 \alpha_1^5 \alpha_2 + 3 \alpha_1^4 \alpha_2^2 + 26 \alpha_1^3 \alpha_2^3 + 3 \alpha_1^2 \alpha_2^4 - 12 \alpha_1 \alpha_2^5 - 4 \alpha_2^6 $$
		
		which is exactly $\Delta_f$ as above.
		
		So, given a cubic polynomial in the general form $f(x) = ax^3 + bx^2 + cx + d$, we can find the discriminant by evaluating
		$$\Delta_f = 4 p^3 + 27 q^2 $$
		
		where as shown in lemma \ref{depressed}, $p$ and $q$ are given in terms of $a$, $b$, $c$, and $d$ by
		$$ p = \frac{3ac-b^2}{3a^2} \ , \ q = \frac{2b^3 - 9abc + 27a^2d}{27a^3} . $$
	\end{proof}
	
	
	
	
\end{document}