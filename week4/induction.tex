\documentclass[12pt]{amsart}

\usepackage{amsthm}
\usepackage{amssymb}
\usepackage{amsmath}
\usepackage{geometry}
\geometry{a4paper}

\newtheorem{thm}{Theorem}
\newtheorem{prop}{Proposition}
\newtheorem{lem}{Lemma}
\newtheorem{cor}{Corollary}
\newtheorem{rem}{Remark}
\newtheorem{exa}{Example}
\newtheorem{clm}{Claim}

\newtheoremstyle{case}{}{}{}{}{}{:}{ }{}
\theoremstyle{case}
\newtheorem{case}{Case}

\title{Induction}
\author{Jason Medcoff}

\begin{document}
	
	\maketitle
	
	\noindent \textbf{Problem 1.} Prove by induction that $7^n - 1$ is divisible by 6 for all $n \geq 1$.
	
	\begin{proof}
		First, let $n = 1$. Then we have
		$$7^1 - 1 = 6 . $$
		Clearly, 6 is divisible by 6, so the statement holds for $n=1$.
		
		Next, let $n=k$. Assuming the statement holds, we have
		$$$$
		\begin{equation*}
		\begin{split}
		7^k - 1 & = 6a \\
		7^k & = 6a + 1
		\end{split}
		\end{equation*}
		for some integer a. Letting $n = k+1$, we can show by algebraic manipulation that
		\begin{equation*}
		\begin{split}
		7^{k+1} & = 7(6a + 1) \\
		7^{k+1} & = 42a + 7 \\
		7^{k+1} - 1 & = 42a + 6 .\\
		\end{split}
		\end{equation*}
		We can factor the right side of the equality to obtain
		$$ 6(7a + 1) ,$$
		and then let $b = 7a + 1$. Now we have
		$$ 7^{k+1} - 1 = 6b $$
		for some integer $b$, and therefore $7^{k+1} - 1$ is divisible by 6.
	\end{proof}
	
	\noindent \textbf{Problem 2.} Show that, given $x_1 = 1$ and
	$$ x_{k+1} = \frac{x_k}{x_k + 2} , $$
	we can write
	$$ x_n = \frac{1}{2^n - 1} $$
	for all $n \geq 1$.
	
	\begin{proof}
		Begin by calculating the first few values of $x_n$: namely, $x_i$ for $i \in \{2,3,4,5\}$.
		We can see that
		\begin{equation*}
		\begin{split}
		x_2 & = \frac{1}{4-1} = \frac{1}{3}, \\ x_3 & = \frac{1}{8-1} = \frac{1}{7}, \\ x_4 & = \frac{1}{16-1} = \frac{1}{15}, \\ x_5 & = \frac{1}{32-1} = \frac{1}{31} .
		\end{split}
		\end{equation*}
		
		Let $n=1$. Then we have
		$$x_1 = \frac{1}{2-1} = 1$$
		which is as given.
		
		Now let $n=k$. Then we can write
		$$ x_k = \frac{1}{2^k - 1} = \frac{x_{k-1}}{x_{k-1}+2}.$$
		Letting $n=k+1$, we have
		$$ x_{k+1} = \frac{x_k}{x_k + 2} = \frac{\frac{1}{2^k-1}}{\frac{1}{2^k - 1} + 2} . $$
		Writing the denominator as
		$$\frac{1+2(2^k-1)}{2^k-1} , $$
		we obtain
		\begin{equation*}
		\begin{split}
		x_{k+1} & = \Big( \frac{1}{2^k -1} \Big) \Big( \frac{2^k-1}{1+2(2^k-1)} \Big) \\
		& = \frac{1}{1+2(2^k-1)} \\
		& = \frac{1}{1 + 2^{k+1} - 2} \\
		& = \frac{1}{2^{k+1} - 1}
		\end{split}
		\end{equation*}
		So, the statement holds for $x_{k+1}$.
		
	\end{proof}
	
	
	
	
	
	
	
	
	
	
	
	
	
	
	
	
	
	
	
	
	
	
	
	
	
	
\end{document}