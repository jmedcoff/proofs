\message{ !name(paper.tex)}
\documentclass[12pt]{amsart}
\usepackage{geometry} % see geometry.pdf on how to lay out the page. There's lots.
\geometry{a4paper} % or letter or a5paper or ... etc
% \geometry{landscape} % rotated page geometry

\newtheorem{thm}{Theorem}
\newtheorem{prop}{Proposition}
\newtheorem{lem}{Lemma}

\newtheorem{cor}{Corollary}

\newtheorem{rem}{Remark}
\newtheorem{exa}{Example}

% See the ``Article customise'' template for come common customisations

\title{How I fell in love with mathematics}
\author{Karl Gauss}
\date{} % delete this line to display the current date






%%% BEGIN DOCUMENT
\begin{document}

\message{ !name(paper.tex) !offset(-3) }





\begin{abstract}
In this short article ....
\end{abstract}


\maketitle
\tableofcontents

\section{Introduction}


%*************************
\section{The early years}

I was the son of an account who worked for a mining company. ...


\subsection{Elementary Scool}

When I was five years old, my parents decided that I was old enough to go to school. 




\section{College education and mathematics}




\section{Being a mathematician}


My first and most important theorem was 

\begin{thm}\label{thm-1}
Let $f(x)$ be a polynomial of degree $n\geq 2$ with complex coefficients.  Then $f(x)$ has exactly $n$ roots. 
\end{thm}

\proof
I will leave the proof as an exercise. 


\qed

A first proof of this theorem appeared in \cite{gauss-1}

Based on Thm.~\ref{thm-1} we can conclude that .... 

%*****************
\begin{thebibliography}{50}

 
\bibitem{gauss-1} Gauss, Carl Friedrich
Disquisitiones arithmeticae.

\end{thebibliography}


\end{document}
\message{ !name(paper.tex) !offset(-95) }
