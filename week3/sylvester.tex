\documentclass[12pt]{amsart}

\usepackage{amsthm}
\usepackage{amssymb}
\usepackage{amsmath}
\usepackage{geometry}
\usepackage{enumitem}
\geometry{a4paper}

\newtheorem{thm}{Theorem}
\newtheorem{prop}{Proposition}
\newtheorem{lem}{Lemma}
\newtheorem{cor}{Corollary}
\newtheorem{rem}{Remark}
\newtheorem{exa}{Example}
\newtheorem{clm}{Claim}

\newtheoremstyle{case}{}{}{}{}{}{:}{ }{}
\theoremstyle{case}
\newtheorem{case}{Case}

\title{Double Roots and Sylvester Matrices}
\author{Jason Medcoff}

\begin{document}
	
	\maketitle
	
	\section{Presence of a Double Root}
	
	Let $f(x)$ be a degree 2 polynomial with real coefficients given by
	$$ f(x) = ax^2 + bx + c . $$
	Without loss of generality, assume $a$ is nonzero.
	
	\begin{clm}
		The following are equivalent:
		\begin{enumerate}[label=(\roman*)]
			\item $f(x)$ has a double root $\alpha$
			\item $f'(\alpha) = 0$
			\item $\Delta_f = 0$
			\item $f(x)$ has the same sign as the leading coefficient $a \in \mathbb{R}$, for all $x \in \mathbb{R} \setminus \{\alpha\}$.
		\end{enumerate}
	\end{clm}
	
	\begin{proof}
		In (1) we will show that $\mathit{(i)}$ implies $\mathit{(ii)}$, and (2) shows the reverse. In (3) it is shown that $\mathit{(i)}$ implies $\mathit{(iii)}$ and in (4) the reverse is shown. In (5) and (6) it is demonstrated that $\mathit{(i)}$ implies $\mathit{(iv)}$ and $\mathit{(iv)}$ implies $\mathit{(i)}$, respectively.
		
		
		\begin{enumerate}
			\item % 1 implies 2
			The quadratic polynomial $f(x)$ can be written as
			$$ f(x) = a(x-\alpha_1)(x-\alpha_2) $$
			but in the case of a double root, this simplifies to
			$$ f(x) = a(x-\alpha)(x-\alpha) = a(x-\alpha)^2 . $$
			Consider the derivative $f'(x)$. Then we have
			\begin{equation*}
			\begin{split}
			f'(x) & = \frac{d}{dx} a(x-\alpha)^2 \\
			& = 2a(x-\alpha) .
			\end{split}
			\end{equation*} 
			It follows that if we evaluate $f'(\alpha)$, we have
			$$ f'(\alpha) = 2a(\alpha - \alpha) = 2a(0) = 0 . $$
			
			
			$\newline$
			\item % 2 implies 1
%			It cannot be proven that $\mathit{(ii)}$ implies anything. $\alpha$ is only given any meaning in $\mathit{(i)}$ as a double root; otherwise we do not know anything about it, so we must assume it is any number on $f$'s domain. But then there is nothing special about $f'(\alpha)$ being zero, since every quadrilateral polynomial has some value for which the derivative at that value is zero. Not enough information is given in $\mathit{(ii)}$ to show any of the other assumptions.

			Given that $\alpha$ is a root of $f(x)$, we want to show that $f'(\alpha) = 0$ implies that $\alpha$ is a double root. Suppose instead that $f(x)$ has two distinct roots, $\alpha_1$ and $\alpha_2$. Then we can write
			$$ f(x) = a(x-\alpha_1)(x-\alpha_2) . $$
			Taking the derivative, we have
			\begin{equation*}
			\begin{split}
			f'(x) & = a[(x-\alpha_1) + (x-\alpha_2)] \\
			& = a(2x - \alpha_1 - \alpha_2) .
			\end{split}
			\end{equation*}
			If we compute $f'(\alpha_1)$ and $f'(\alpha_2)$, we have
			$$ f'(\alpha_1) = a(\alpha_1 - \alpha_2), \ f'(\alpha_2) = a(\alpha_2 - \alpha_1) $$
			and since we assumed that $\alpha_1$ and $\alpha_2$ are distinct and $a$ is nonzero, we have $f'(\alpha_1) \neq 0$ and $f'(\alpha_2) \neq 0$. Then, by the contrapositive, we have that if $\alpha$ is a root of $f(x)$, $f'(\alpha) = 0$ implies $\alpha$ is a double root of $f(x)$.
			
			
			
			
			
			
			$\newline$
			\item % 1 implies 3
			we write the discriminant as
			$$ \Delta_f = (\alpha_1 - \alpha_2)(\alpha_2 - \alpha_1) . $$
			If $f(x)$ has a double root, then $\alpha_1 = \alpha_2$, and we have
			$$ \Delta_f = (\alpha_1 - \alpha_1)(\alpha_1 - \alpha_1) $$
			and it is shown that $\Delta_f = 0$.
			
			
			$\newline$
			\item % 3 implies 1
			With the discriminant given as 
			$$ \Delta_f = (\alpha_1 - \alpha_2)(\alpha_2 - \alpha_1) , $$
			it must be that if $\Delta_f = 0$, we have
			$$ 0 = (\alpha_1 - \alpha_2)(\alpha_2 - \alpha_1) $$
			and it follows that $\alpha_1$ must equal $\alpha_2$. Then we can write the polynomial $f(x)$ as
			$$ f(x) = a(x- \alpha_1)(x-\alpha_2) $$
			and since $\alpha_1 = \alpha_2$ we have
			$$ f(x) = a(x-\alpha_1)^2 . $$
			Then the root $\alpha_1$ has multiplicity two, which by definition, makes it a double root.
			
			$\newline$
			\item % 1 implies 4
			As above, we can write the quadratic polynomial as
			$$ f(x) = a(x-\alpha)^2 . $$
			Consider the term $(x-\alpha)^2$. As this term is squared, it is always nonnegative. So, $a$ is the only term in the expression that determines the sign of $f(x)$. It follows that for positive $a$, we have a positive term $a$ multiplied by a positive term $(x-\alpha)^2$ which gives a positive product. For negative $a$, we have a negative term multiplied by a positive term, yielding a negative product. 
			
			The only case where this is not true is when $x = \alpha$, so we can say that $f(x)$ has the same sign as the leading coefficient $a$ for all $x \in \mathbb{R} \setminus \{\alpha\}$.
			
			$\newline$
			\item % 4 implies 1
			If $f(x)$ has the same sign as its leading coefficient $a$ for all $x \in \mathbb{R} \setminus \{\alpha\}$, we want to show that $f(x)$ has a double root.
			Suppose instead that $f(x)$ has two unique roots. Assume without loss of generality that $\alpha_1 < \alpha_2$. Then if we write
			$$ f(x) = a(x-\alpha_1)(x-\alpha_2) $$
			we can see that there would exist some values of $x$ such that 
			$$\alpha_1 < x < \alpha_2$$
			and we would have
			$$ x - \alpha_1 > 0, \ \ x - \alpha_2 < 0 $$
			and so the product $(x-\alpha_1)(x-\alpha_2)$ would be negative. This would lead to $f(x)$ not having the same sign as $a$ for $\alpha_1 < x < \alpha_2$.
			
			This contradicts our assumption that $f(x)$ has the same sign as its leading coefficient $a$ for all $x \in \mathbb{R} \setminus \{\alpha\}$, so it must be the case that $\alpha_1 = \alpha_2$ and $f(x)$ has a double root.
			
			$\newline$
			Each statement has been proven true if and only if $\mathit{(i)}$ is true. Therefore, all four statements are equivalent.			
		\end{enumerate}
	\end{proof}
	
	\section{Sylvester Matrices}
	
	\subsection{Problem 2}
	Let
	\begin{equation*}
	\begin{split}
	f(x) & = x^5 - 3x^4 - 2x^3 + 3x^2 + 7x + 6 \\
	g(x) & = x^4 + x^2 + 1 .
	\end{split}
	\end{equation*}
	Then the Sylvester matrix of $f$ and $g$ is
	\[
	\begin{bmatrix}
	1 & 0 & 0 & 0 & 1 & 0 & 0 & 0 & 0\\
	-3 & 1 & 0 & 0 & 0 & 1 & 0 & 0 & 0\\
	-2 & -3 & 1 & 0 & 1 & 0 & 1 & 0 & 0\\
	3 & -2 & -3 & 1 & 0 & 1 & 0 & 1 & 0\\
	7 & 3 & -2 & -3 & 1 & 0 & 1 & 0 & 1\\
	6 & 7 & 3 & -2 & 0 & 1 & 0 & 1 & 0\\
	0 & 6 & 7 & 3 & 0 & 0 & 1 & 0 & 1\\
	0 & 0 & 6 & 7 & 0 & 0 & 0 & 1 & 0\\
	0 & 0 & 0 & 6 & 0 & 0 & 0 & 0 & 1\\
	\end{bmatrix}
	\]
	$Res(f,g,x)$ is the determinant of this matrix. Computing the determinant, we have
	$$ Res(f,g,x) = 0 . $$
	
	\subsection{Problem 3}
	Consider
	$$ f(x) = 6x^4 - 23x^3 - 19x + 4 . $$
	$f(x)$ has multiple roots if the discriminant $Res(f,f',x)$ is shown to be zero.
	We can compute $f'(x)$ to be
	$$ 24x^3 - 69x^2 - 19 . $$
	The Sylvester matrix of $f$ and $f'$ is
	\[
	\begin{bmatrix}
	6 & 0 & 0 & 24 & 0 & 0 & 0\\
	-23 & 6 & 0 & -69 & 24 & 0 & 0\\
	0 & -23 & 6 & 0 & -69 & 24 & 0\\
	-19 & 0 & -23 & -19 & 0 & -69 & 24\\
	4 & -19 & 0 & 0 & -19 & 0 & -69\\
	0 & 4 & -19 & 0 & 0 & -19 & 0\\
	0 & 0 & 4 & 0 & 0 & 0 & -19\\
	\end{bmatrix}
	\]
	and the determinant of this matrix is 
	$$Res(f,f',x) = -3921998592 \neq 0 .$$
	So, $f(x)$ does not have any multiple roots.
	
	\subsection{Problem 4}
	Consider
	$$ f(x) = x^4 - bx + 1 . $$
	We want $b$ such that $f(x)$ has a double root. For $f$ to have a double root we need
	$$\frac{-1^\frac{4(3)}{2}}{1}Res(f,f',x) = (1) Res(f,f',x) = 0 . $$
	The Sylvester matrix of $f$ and $f'$ is
	\[
	\begin{bmatrix}
	1 & 0 & 0 & 4 & 0 & 0 & 0\\
	0 & 1 & 0 & 0 & 4 & 0 & 0\\
	0 & 0 & 1 & 0 & 0 & 4 & 0\\
	-b & 0 & 0 & -b & 0 & 0 & 4\\
	1 & -b & 0 & 0 & -b & 0 & 0\\
	0 & 1 & -b & 0 & 0 & -b & 0\\
	0 & 0 & 1 & 0 & 0 & 0 & -b\\
	\end{bmatrix}
	\]
	and the determinant gives
	$$ Res(f,f',x) = -27b^4 + 256 . $$
	Using this in the above equation, we have
	$$ (1) (-27b^4 + 256) = 0 $$
	and it follows that
	$$27b^4 = 256 . $$
	So, for all $b$ that satisfy
	$$ b^4 = \frac{256}{27} , $$
	$f(x)$ will have a double root.
	
	
	
	\subsection{Problem 5}
	Consider
	$$ f(x) = x^3 - px + 1 . $$
	We want $p$ such that $f(x)$ has a double root. So we need
	$$\frac{-1^\frac{3(2)}{2}}{1}Res(f,f',x) = (-1) Res(f,f',x) = 0 . $$
	The Sylvester matrix of $f$ and $f'$ is
	\[
	\begin{bmatrix}
	1 & 0 & 3 & 0 & 0\\
	0 & 1 & 0 & 3 & 0\\
	-p & 0 & -p & 0 & 3\\
	1 & -p & 0 & -p & 0\\
	0 & 1 & 0 & 0 & -p\\
	\end{bmatrix}
	\]
	and computing the determinant yields
	$$ Res(f,f',x) = -4p^3 + 27 . $$
	Setting this equal to zero, we have
	$$ 0 = -4p^3 + 27 $$
	and it follows that
	$$ 4p^3 = 27 . $$
	So for all $p$ such that
	$$ p^3 = \frac{27}{4}$$
	is satisfied, $f(x)$ will have a double root.
	
	
	
	
	
	
	
	
	
	
	
\end{document}