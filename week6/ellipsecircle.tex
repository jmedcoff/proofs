\documentclass[12pt]{amsart}

\usepackage{amsthm}
\usepackage{amssymb}
\usepackage{amsmath}
\usepackage{geometry}
\geometry{a4paper}

\newtheorem{thm}{Theorem}
\newtheorem{prop}{Proposition}
\newtheorem{lem}{Lemma}
\newtheorem{cor}{Corollary}
\newtheorem{rem}{Remark}
\newtheorem{exa}{Example}
\newtheorem{clm}{Claim}

\newtheoremstyle{case}{}{}{}{}{}{:}{ }{}
\theoremstyle{case}
\newtheorem{case}{Case}

\title{Circles and Ellipses}
\author{Jason Medcoff}

\begin{document}
	\maketitle
	
	%%%%%%%%%%%%%%%%%%%%%%%%%%%%%%%% Problem 1a %%%%%%%%%%%%%%%%%%%%%%%%%%%%%%%%%%%%%%
	
	\noindent \textbf{Problem 1a.} Find the perimeter of
	$$ x^2 + y^2 = r^2 . $$
	
	We know the arc length formula for polar coordinates is
	
	$$ C = \int_{\alpha}^{\beta} \sqrt{ r^2 + \bigg( \frac{dr}{d\theta} \bigg)^2 } d\theta , $$
	
	and in this case we have a constant $r$. So the integral becomes
	
	$$ C = \int_{0}^{2\pi} \sqrt{ r^2 + 0^2 } d\theta = \int_{0}^{2\pi} r d\theta , $$
	
	which evaluates to
	
	$$ C = r(2\pi - 0) = 2\pi r . $$
	
	%%%%%%%%%%%%%%%%%%%%%%%%%%%%%%%% Problem 1b %%%%%%%%%%%%%%%%%%%%%%%%%%%%%%%%%%%%%%
	
	\noindent \textbf{Problem 1b.} Find the area of
	$$ x^2 + y^2 = r^2 . $$
	
	From above, we know that the perimeter is given by $C = 2\pi r$. We can use the shell method of integration to partition the circle into thin rings of radius $s$ and width $ds$. Then the area of each ring is simply $2\pi s ds$, and we have
	
	$$ A = \int_{0}^{r} 2\pi s ds . $$
	
	Integrating gives
	
	$$ A = 2\pi \frac{s^2}{2} \biggr\rvert^{r}_{0} = \pi r^2 - 0 = \pi r^2 . $$
	$\newline$
	
	%%%%%%%%%%%%%%%%%%%%%%%%%%%%%%%% Problem 2a %%%%%%%%%%%%%%%%%%%%%%%%%%%%%%%%%%%%%%
	
	\noindent \textbf{Problem 2a.} Find the perimeter of
	$$ \frac{x^2}{a^2} + \frac{y^2}{b^2} = 1 . $$
	
	
	We know that the general formula for arc length of a parametric curve is given by
	$$ p = \int_{c}^{d} \sqrt{ \Big( \frac{dx}{dt} \Big) ^2 + \Big( \frac{dy}{dt} \Big) ^2} dt . $$	
	The given equation can be parameterized such that $x = a\cos\theta$ and $y = b\sin\theta$. In addition, finding the arc length in the first quadrant and then multiplying by four will give the total circumference. Then we have the circumference of the curve as
	$$ p = 4 \int_{0}^{\frac{\pi}{2}} \sqrt{ a^2\sin^2\theta + b^2\cos^2\theta } d\theta $$
	Noting that $\sin^2\theta = 1 - cos^2\theta$, we can write
	$$ p = 4 \int_{0}^{\frac{\pi}{2}} \sqrt{ a^2(1 - cos^2\theta) + b^2\cos^2\theta } d\theta $$
	and simplify to get
	$$ p = 4 \int_{0}^{\frac{\pi}{2}} \sqrt{ a^2 + (b^2 - a^2)cos^2\theta } d\theta . $$
	We let the eccentricity $\varepsilon$ be defined as
	$$ \varepsilon = \frac{\sqrt{a^2-b^2}}{a} . $$
	Now we can write the circumference as
	$$ p = 4a \int_{0}^{\frac{\pi}{2}} \sqrt{ 1 - \varepsilon^2\cos^2\theta } d\theta . $$
	
	We know that the binomial series, a special Maclaurin series, gives
	$$ (z+1)^\alpha = \sum_{k=0}^{\infty} \binom{\alpha}{k} z^k = 1 + \alpha z + \frac{\alpha(\alpha - 1)}{2!} z^2 + \dots $$
	This series can be applied to the argument of the circumference integral with $\alpha = 1/2$ and $z = -\varepsilon^2\cos^2\theta$. Specifically, we have
	$$ \sqrt{ 1 - \varepsilon^2\cos^2\theta } = 1 - \frac{\varepsilon^2\cos^2\theta}{2} - \sum_{k=2}^{\infty} \binom{1/2}{k} (\varepsilon^2 \cos^2\theta)^k $$
	
	Note that since $\varepsilon$ describes the eccentricity of the ellipse, its value must be strictly between 0 and 1. Also, $|\cos^2\theta| \leq 1$ for all $\theta$. So, the series converges.
	
	Returning to the circumference, we have
	
	\begin{equation*}
	\begin{split}
	p & = 4a \int_{0}^{\frac{\pi}{2}} \Bigg[ 1 - \frac{\varepsilon^2\cos^2\theta}{2} - \sum_{k=2}^{\infty} \binom{1/2}{k} z^k \Bigg] d\theta \\
	& = 4a \int_{0}^{\frac{\pi}{2}} d\theta - 4a \int_{0}^{\frac{\pi}{2}} \frac{\varepsilon^2\cos^2\theta}{2} d\theta - 4a \int_{0}^{\frac{\pi}{2}} \sum_{k=2}^{\infty} \binom{1/2}{k} (\varepsilon^2 \cos^2\theta)^k d\theta
	\end{split}
	\end{equation*}
	
	The first two terms are integrated easily. With regards to the integral of the summation, it is clear that the quantity must be finite, as we are calculating a perimeter. So we can apply Fubini's Theorem to exchange the integral and the summation to obtain
	
	$$ p = 4a \int_{0}^{\frac{\pi}{2}} d\theta - 4a \int_{0}^{\frac{\pi}{2}} \frac{\varepsilon^2\cos^2\theta}{2} d\theta - 4a \sum_{k=2}^{\infty} \int_{0}^{\frac{\pi}{2}} \binom{1/2}{k} (\varepsilon^2 \cos^2\theta)^k d\theta . $$
	
	Now we can rearrange the placement of terms, yielding a much nicer looking integral:
	
	$$ p = 4a \int_{0}^{\frac{\pi}{2}} d\theta - 4a \int_{0}^{\frac{\pi}{2}} \frac{\varepsilon^2\cos^2\theta}{2} d\theta - 
	4a \sum_{k=2}^{\infty} \varepsilon^{2k} \binom{1/2}{k} \int_{0}^{\frac{\pi}{2}} \cos^{2k}\theta d\theta . $$
	
	Using Wolfram Alpha to assist with the last integral, we can evaluate the integrals to obtain
	
	$$ p = 2\pi a - 4a \frac{\pi}{4} \frac{\varepsilon^2}{2} - 4a \sum_{k=2}^{\infty} \varepsilon^{2k} \binom{1/2}{k} \frac{\pi}{2} \frac{1\cdot3\cdot5\cdots(2k-1)}{2\cdot4\cdot6\cdots2k} . $$
	
	The last fraction is obtained from Wallis's cosine formula. We can make the observation that
	
	$$ \binom{1/2}{k} = \frac{(-1)^{k+1} 1 \cdot 3 \cdot 5 \cdots (2k-5) \cdot (2k-3)}{k! 2^k} $$
	
	So we can write
	
	
	
	$$ p = 2\pi a - 4a \frac{\pi}{4} \frac{\varepsilon^2}{2} - 4a \sum_{k=2}^{\infty} \varepsilon^{2k} \frac{\pi}{2} \frac{[1\cdot3\cdot5\cdots(2k-3)]^2 (2k-1)}{(k!2^k)^2} .	$$
	
	Finally, we simplify to obtain
	
	$$ p = 2\pi a \bigg[ 1 - \frac{\varepsilon^2}{4} - \sum_{k=2}^{\infty} \varepsilon^{2k} \frac{[1\cdot3\cdot5\cdots(2k-3)]^2 (2k-1)}{(k!2^k)^2}\bigg] . $$
	
	In the case of a circle where $r = a = b$, eccentricity $\varepsilon = 0$ and it is clear from the equation that circumference collapses to $p = 2\pi r$.
	
	%%%%%%%%%%%%%%%%%%%%%%%%%%%%%%%% Problem 2b %%%%%%%%%%%%%%%%%%%%%%%%%%%%%%%%%%%%%%
	
	\noindent \textbf{Problem 2b.} Find the area of 
	$$ \frac{x^2}{a^2} + \frac{y^2}{b^2} = 1 . $$
	
	Note that we can obtain the total area by multiplying the area in the first quadrant by four. 
	Rewriting the equation in terms of $y$, we have
	
	$$ y = b \sqrt{1 - \frac{x^2}{a^2}} $$
	
	and we would like to evaluate
	
	$$ A = 4\int_{0}^{\frac{\pi}{2}} y dx $$
	
	which we can write as
	
	$$ A = 4b \int_{0}^{\frac{\pi}{2}} \sqrt{1 - \frac{x^2}{a^2}} dx . $$
	
	Let $x = a\sin\theta$. Then we have
	
	$$ \frac{dx}{d\theta} = a\cos\theta $$
	
	and we can substitute into the integral to obtain
	
	$$ A = 4ab \int_{0}^{\frac{\pi}{2}} \sqrt{1 - \sin^2\theta} \cos\theta d\theta . $$
	
	We observe that since 
	
	$$1 - \sin^2\theta = cos^2\theta$$
	
	we can write this as
	
	$$ A = 4ab \int_{0}^{\frac{\pi}{2}} \sqrt{\cos^2\theta} \cos\theta d\theta = 4ab \int_{0}^{\frac{\pi}{2}} \cos^2\theta d\theta = 4ab \int_{0}^{\frac{\pi}{2}} \frac{1+\cos2\theta}{2} d\theta $$
	
	and finally integrate, yielding
	
	$$ A = 4ab \frac{1}{2}(\theta+\sin\theta\cos\theta) \biggr\rvert_{0}^{\frac{\pi}{2}} = 2ab(\frac{\pi}{2} - 0) = \pi ab . $$
	
	
\end{document}