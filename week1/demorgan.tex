\documentclass[12pt]{amsart}

\usepackage{amsthm}
\usepackage{amssymb}
\usepackage{amsmath}
\usepackage{geometry}
\geometry{a4paper}

\newtheorem{thm}{Theorem}
\newtheorem{prop}{Proposition}
\newtheorem{lem}{Lemma}
\newtheorem{cor}{Corollary}
\newtheorem{rem}{Remark}
\newtheorem{exa}{Example}

\title{DeMorgan's Laws and Hexagonal Symmetry}
\author{Jason Medcoff}

\begin{document}
	
	\maketitle
	
	\begin{abstract}
		Here proofs for each of DeMorgan's two laws are given. A regular hexagon is constructed with compass and straightedge, and its symmetries are described.
	\end{abstract}
	
	
	\section{DeMorgan's Laws}
	
	\begin{thm}
		For any two sets $A$ and $B$, $(A \cup B)^{\complement} = A^{\complement} \cap B^{\complement}$.
	\end{thm}
	
	\begin{proof}
		
		Let $x \in (A \cup B)^{\complement}$. Then $x \notin A \cup B$. It must be the case that $x \notin A$ and $x \notin B$. Thus, $x \in A^{\complement}$ and $x \in B^{\complement}$, so $x \in A^{\complement} \cap B^{\complement}$. This means that $\forall x \in (A \cup B)^{\complement}$, $x \in A^{\complement} \cap B^{\complement}$.
		By the definition of set inclusion, $(A \cup B)^{\complement} \subset A^{\complement} \cap B^{\complement}$.
		$\newline$
		
		Let $y \in A^{\complement} \cap B^{\complement}$. Then $y \in A^{\complement}$ and $y \in B^{\complement}$. So $y \notin A$ and $y \notin B$. Therefore, $y \notin A \cup B$, so it must be that $y \in (A \cup B)^{\complement}$. Similar to above, it follows that $A^{\complement} \cap B^{\complement} \subset (A \cup B)^{\complement}$.
		$\newline \newline$
		By definition of set equality, $(A \cup B)^{\complement} = A^{\complement} \cap B^{\complement}$.	 		
	\end{proof}
	
	\begin{thm}
		For any two sets A and B, $(A \cap B)^{\complement} = A^{\complement} \cup B^{\complement}$.
	\end{thm}
	
	\begin{proof}
		Let $x \in (A \cap B)^{\complement}$. Then $x \notin A \cap B$. So $x\notin A$ or $x \notin B$. Therefore $x \in A^{\complement}$ or $x \in B^{\complement}$. Thus, $x \in A^{\complement} \cup B^{\complement}$. By the definition of set inclusion, $(A \cap B)^{\complement} \subset A^{\complement} \cup B^{\complement}$.
		$\newline$
		
		Let $y \in A^{\complement} \cup B^{\complement}$. It follows that $y \in A^{\complement}$ or $y \in B^{\complement}$. Then $y \notin A$ or $y \notin B$. So $y \notin A \cap B$, therefore $y \in (A \cap B)^{\complement}$. Similar to above, we find that $A^{\complement} \cup B^{\complement} \subset  (A \cap B)^{\complement}$.
		$\newline$
		By definition of set equality, $(A \cap B)^{\complement} = A^{\complement} \cup B^{\complement}$.
	\end{proof}
	
	
	\section{The Regular Hexagon}
	
	A regular hexagon is a six sided polygon such that all angles are equal in measure and all sides have the same length. One can draw a right triangle inside a regular hexagon to determine that the radius of a circumscribed circle is equal to the side length of the hexagon. Thus, a regular hexagon can be constructed with compass and straightedge by making a circle, then using its radius to draw equally spaced points along its circumference, labeled $\mathit{1}, \mathit{2}, \mathit{3}, \mathit{4}, \mathit{5},$ and $\mathit{6}$.
	
	The regular hexagon as labeled has twelve symmetries: six rotational and six reflectional. Rotating the hexagon counterclockwise by an integer multiple of 60 degrees gives a symmetry, so the six rotational symmetries are 60, 120, 180, 240, 300, and 360 degrees. 
	
	The reflectional symmetries can be obtained by defining the axes of symmetry on the polygon. Three of the axes are constructed by connecting opposite vertices. The remaining three are found by bisecting each side of the hexagon. Thus, the six reflectional symmetries are given by the line segments $\overline{\mathit{14}}, \overline{\mathit{25}}, \overline{\mathit{36}}, \overline{AD}, \overline{BE},$ and $\overline{CF}$.
	
	
	
	
	
	
	
	
	
	
	
	
	
	
\end{document}